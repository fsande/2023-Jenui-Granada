A famous theorem of the mathematician Pierre de Fermat, proved after more than 300 years, 
states that, for any natural number $n \geq 3$, there is no natural solution (except for x=0 or y=0) to the equation
 $x^n + y^n = z^n$

For $n=2$, by contrast, there are infinite non-trivial solutions. 
For instance, $3^2 + 4^2 = 5^2$, $5^2 + 12^2 = 13^2$, $6^2 + 8^2 = 10^2$, ...

Write a C++ program that, given four natural numbers a, b, c, d with $a \leq b$ and $c \leq d$, prints a natural solution to the equation
  $x^2 + y^2 = z^2$
such that $a \leq x \leq b$ and $c \leq y \leq d$.

The program input consists of four natural numbers a, b, c, d such that $a<=b$ and $c<=d$.

The program output should be a line with a natural solution to the equation
  $x^2 + y^2 = z^2$
that fulfills $a \leq x \leq b$ and $c \leq y \leq d$. 

If there is more than one solution, print the one with the smallest x. 

If there is a tie in x, print the solution with the smallest y. 

If there are no solutions, print ``No solution!''.

For example, if the input is:
\texttt{2 5 4 13}

The output should be

\texttt{3} \texttt{\^} \texttt{2 + 4} \texttt{\^} \texttt{2 = 5} \texttt{\^} \texttt{2}

and if the input is:
\texttt{1 1 1 1}

     The output should be 
\texttt{No solution!}
