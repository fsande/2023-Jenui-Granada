\documentclass[twocolumn,twoside,a4paper, 10pt]{article}
\usepackage[utf8]{inputenc}
\usepackage[T1]{fontenc}
\usepackage[spanish]{babel}
\usepackage{balance}
\usepackage{jenuia4}
\usepackage{url}
%\usepackage{graphicx}

% Mis paquetes. No incluídos en la plantilla de Jenui
\usepackage{hyperref}
\usepackage{float} % For float Listings
\usepackage[pdftex]{color}
\usepackage[pdftex]{graphicx}
%\graphicspath{{FIGURES/png/}} 
%%%%%%%%%%%%%%%%%%%%%%%%%%%%%%%%%%%%%%%%%%%%%%%%%%%%%%%%%%%%%%%%%%%%%%%%%%%%%%%%%%%%%%%%%%%%%
\definecolor{marron}       {rgb}{0.496, 0.203, 0.152}
\definecolor{verde-claro}  {rgb}{0.625, 0.734, 0.199}
\definecolor{oscuro}       {rgb}{0.187, 0.141, 0.285}
\definecolor{gris}     	   {rgb}{0.500, 0.500, 0.500}
\definecolor{bgd-listings} {rgb}{0.999, 0.999, 0.900}
\definecolor{gray97}{gray}{.97}
\definecolor{gray75}{gray}{.75}
\definecolor{gray45}{gray}{.45}
%%%%%%%%%%%%%%%%%%%%%%%%%%%%%%%%%%%%%%%%%%%%%%%%%%%%%%%%%%%%%%%%%%%%%%%%%%%%%%%%%%%%%%%%%%%%%
%% Para listados de código
\usepackage{listings,multicol}  % <--- multicol only required, if the multicols= option shall be used
\lstset{language=C++,
        basicstyle=\scriptsize\ttfamily,
        commentstyle=\color{blue},           % comentarios en azul
        keywordstyle=\bfseries,
        stringstyle=\color{red}\ttfamily,
        morecomment=[l][\color{magenta}]{\#}
        framexleftmargin=5mm,									% BGD: Margen izquierdo de los marcos (para que quepan num. de línea) 
        frame=trbl,														% BGD: Cuadro  (t/T: top, r/R: right, b/B: bottom, l/L: left): x->thin, X->thick
        framesep=4pt,
}
%\usepackage{listings} 
%\lstloadlanguages{C++}
%\lstset{
%  language=C++,                        % C
%  keywordstyle=\color{red},            % Palabras clave en rojo
%  identifierstyle=\ttfamily,
%  commentstyle=\color{blue},           % comentarios en azul
%  stringstyle=\color{green},           % cadenas en verde
%	%showstringspaces=false
%  language=C,                          	% C
%  %backgroundcolor=\color{bgd-listings},
%	%backgroundcolor=\color{yellow},     	% Códigos sobre fondo amarillo
%	%frame=lines,                        	% Línea arriba y abajo de cada listado de código
%  %framerule=0pt,
%  %basicstyle=\small\ttfamily,
%	basicstyle=\scriptsize,                   	% Listados en tiny
%	captionpos=b,													% Posición de los títulos abajo (bottom)
%  keywordstyle=\bfseries,
%	%keywordstyle=\color{black}\textbf,   	% BGD: Palabras clave en negro y negrita
%	identifierstyle=\color{black}\ttfamily, % BGD: Identificadores en negro	
%	commentstyle=\color{blue}\textit,  	 	% BGD: Comentarios en azul cursiva
%  stringstyle=\ttfamily,
%	%stringstyle=\color{red}\texttt,       % BGD: Cadenas en rojo
%	directivestyle=\color{oscuro}\texttt,	
%  emph={pragma,llc,omp},								% BGD: Resaltar las palabras pragma, llc y omp
%	emphstyle=\textbf,
%	showstringspaces=false,								% BGD: No mostar espacios
%	columns=fixed,
%	basewidth={0.6em, 0.45em},
%	xleftmargin=5mm,
%	keepspaces=true,											% BGD: Mantener espacios
%	framexleftmargin=5mm,									% BGD: Margen izquierdo de los marcos (para que quepan num. de línea) 
%	frame=trbl,														% BGD: Cuadro  (t/T: top, r/R: right, b/B: bottom, l/L: left): x->thin, X->thick
%	framesep=4pt,
%	%frameround=fttt,											% BGD: Cuadro con esquina curvadas
%  %framextopmargin=3pt,
%  %framexbottommargin=3pt,
%  %framexleftmargin=0.4cm,
%  %framesep=0pt,
%	rulesepcolor=\color{black},						% BGD: Color de los cuadros azul
%	numbers=left,													% BGD: Números de línea a la izquierda
%	firstnumber=1,												% BGD: Número línea empieza en 0 (línea vacía, no se muestra)
%	stepnumber=1,													% BGD: Números línea a línea
%	numberstyle=\scriptsize,										% BGD: Números de línea pequeños
%	numbersep=8pt,												% BGD: Números de línea separados 8 pts a la izquierda
%	numberblanklines=false,								% BGD: No pone números de línea en líneas vacías
%	extendedchars=true,									  % BGD: Utiliza caracteres extendidos (tildes, etc.)
%	inputencoding=latin1,									
%  %inputencoding=utf8x,									
%	tabsize=2,														% BGD: Tamaño de tabulador para indentaciones = 2
%	breaklines=true,											% BGD: Cortar líneas si son muy grandes
%	breakautoindent=true,									% BGD: Autoindentar líneas cortadas
%	postbreak=\space											% BGD: Cortar líneas por espacios
%}

%% minimizar fragmentado de listados
%\lstnewenvironment{listing}[1][]
%   {\lstset{#1}\pagebreak[0]}{\pagebreak[0]}
%\lstdefinestyle{consola}
%   {basicstyle=\scriptsize\bf\ttfamily,
%    backgroundcolor=\color{gray75},
%   }
%\lstdefinestyle{C}
%   {language=C,
%   }
%%%%%%%%%%%%%%%%%%%%%%%%%%%%%%%%%%%%%%%%%%%%%%%%%%%%%%%%%%%%%%%%%%%%%%%%%%%%%%%%%%%%%%%%%%%%%
\AtBeginDocument{
  %\renewcommand\thelstlisting{\arabic{chapter}.\arabic{section}}
  \renewcommand{\thelstlisting}{\arabic{lstlisting}}
}
\renewcommand{\lstlistingname}{Listado} % Los títulos de los códigos insertados se denotan con "Listado"
%%%%%%%%%%%%%%%%%%%%%%%%%%%%%%%%%%%%%%%%%%%%%%%%%%%%%%%%%%%%%%%%%%%%%%%%%%%%%%%%%%%%%%%%%%%%%
\newcommand{\jutge}{\textit{Jutge.org}{}}           % BGD: Nuevos comandos para imprimir con estilo
\newcommand{\ChatGPT}{\textit{ChatGPT}{}}           % BGD: Nuevos comandos para imprimir con estilo
%%%%%%%%%%%%%%%%%%%%%%%%%%%%%%%%%%%%%%%%%%%%%%%%%%%%%%%%%%%%%%%%%%%%%%%%%%%%%%%%%%%%%%%%%%%%%
\title{El impacto de asistentes basados en IA en la enseñanza-aprendizaje de la programación}
%\author{\normalsize 
%\begin{tabular}{@{\extracolsep{3mm}}cc}
%{\large Joe Miró Julià }                  & {\large Mercedes Marqués}\\
%Departament de Matemàtiques i Informàtica & Departamento de Ing. y Ccia. de los Comput.\\
%Universitat de les Illes Balears          & Universitat Jaume I\\
%07122 Palma de Mallorca                   & Castellón\\
%\url{joe.miro@uib.es}                     & \url{merche.marques@uji.es}
%\end{tabular}
%}

\author{ \small
\begin{tabular}{@{\extracolsep{3mm}}c}
\large Anónimo\\
Departamento Anónimo\\
Universidad Anónima\\
XXXXX Ciudad Anónima \\
anonimo@xxxxxx.es
\end{tabular}
}
%\author{ \small
%\begin{tabular}{@{\extracolsep{3mm}}c}
%\large Francisco de Sande \\
%Departamento de Ingeniería Informática y de Sistemas \\
%Universidad de La Laguna \\
%38200 La Laguna. S/C de Tenerife \\
%fsande@ull.es
%\end{tabular}
%}

\date{}

%%%  Referencias
% https://aihub.csic.es/inteligencia-artificial-en-educacion-golem-creativo-o-destructor/

\begin{document}
\maketitle
\thispagestyle{empty}

\begin{abstract}
\noindent En los últimos años, las capacidades de los asistentes de programación basados en Inteligencia 
Artificial se han incrementado de forma muy notable, habiendo trascendido el ámbito de las Ciencias de la Computación.
El impacto que estos sistemas tienen en el campo de la educación y específicamente en la enseñanza de la
programación de ordenadores es particularmente significativo porque se trata del dominio en el que
posiblemente obtienen sus mejores resultados.
De estos avances se han hecho eco medios de comunicación generalistas suscitando un debate muy interesante y
pertinente con una pregunta que atraviesa toda la discusión: ¿suponen estos asistentes el fin de la
programación de ordenadores de la forma en la que la concebimos actualmente?.
En este trabajo se analiza el posible impacto de esta tecnología en el proceso de enseñanza-aprendizaje de la
programación en una asignatura de primer curso del grado en Ingeniería Informática.
Se relacionan algunos experimentos realizados con \ChatGPT{} aplicados a los ejercicios prácticos de
programación que se utilizan como prácticas de laboratorio en la asignatura.
El trabajo finaliza con la presentación de algunas conclusiones que se derivan de los experimentos
realizados así como con el planteamiento de diversas cuestiones que siendo de incuestionable actualidad carecen
de respuestas concluyentes en muchos casos.
\end{abstract}

%%%%%%%%%%%%%%%%%%%%%%%%%%%%%%% Section %%%%%%%%%%%%%%%%%%%%%%%%%%%%%%% 
\section*{Abstract}
\noindent In recent years, the capabilities of AI-based programming assistants have increased dramatically 
and have transcended the field of Computer Science.
The impact that these systems have on the field of education and in particular on the teaching of computer 
programming is particularly significant because this is the field in which they possibly achieve their best 
results.
These advances have been covered in the general media, giving rise to a very interesting and pertinent debate 
with a question that runs through the whole discussion: do these assistants represent the end of computer 
programming as we currently conceive it?.
This paper analyses the possible impact of this technology on the teaching-learning process of programming in 
a first year subject of a Computer Engineering degree.
Some experiments carried out with \ChatGPT{} applied to the practical programming exercises used as laboratory 
practice in the subject are presented.
The work concludes with the presentation of some conclusions derived from the experiments carried out, as well 
as with the posing of several questions that, although undoubtedly topical, lack conclusive answers in many cases.
%%%%%%%%%%%%%%%%%%%%%%%%%%%%%%% Section %%%%%%%%%%%%%%%%%%%%%%%%%%%%%%% 
\section*{Palabras clave}
\noindent Programación, Inteligencia Artificial, Asistentes, ChatGPT, Enseñanza, Aprendizaje

%%%%%%%%%%%%%%%%%%%%%%%%%%%%%%% Section %%%%%%%%%%%%%%%%%%%%%%%%%%%%%%% 
\section{Introducción}
La programación de ordenadores es una actividad transversal a cualquier rama de la informática y su
importancia es compartida por cualquier especialización en esta titulación.
\textit{Informática} 
%\textit{Informática Básica} 
(ASIG, de ahora en adelante) 
es una asignatura de 6 créditos que se imparte en el primer cuatrimestre 
del primer curso del Grado en Ingeniería Informática en la Escuela Anónima
%Superior de Ingeniería y Tecnología de la 
de la Universidad de Anónima.
%La Laguna.
Se trata de la primera asignatura (y única en ese primer cuatrimestre) de perfil eminentemente informático que
cursa el alumnado de la titulación y es en ASIG donde los estudiantes toman
contacto con la programación y en la que han de aprender los fundamentos de esta materia.
Los conceptos objeto de estudio son comunes, con pequeñas variaciones, a cualquier lenguaje de programación
orientado a objetos, que es el paradigma estudiado en ASIG.
El número de estudiantes de primer curso está en torno a los 250 y los contenidos de la asignatura pueden 
consultarse en la Guía Docente
%\cite{ULL:2022:GD} 
\footnote{Guía Docente de ASIG\\ \href{https://www.anonimo.es/}{\scriptsize{\texttt{https://www.anonimo.es/}}}}.
%\footnote{Guía Docente de ASIG\\ \href{https://www.ull.es/apps/guias/guias/view_guide/34182/}{\scriptsize{\texttt{https://www.ull.es/apps/guias/guias/view\_guide/34182/}}}}.
de la misma.
Al margen de tres temas dedicados a una introducción a los Sistemas Operativos, Redes y Bases de Datos,
el grueso de los contenidos (en torno a 12 de las 15 semanas del cuatrimestre) se dedican a introducir al
alumnado en la programación, siendo C++ el lenguaje vehicular utilizado para estudiarla.
Se trata de una asignatura con una importante proporción de contenidos prácticos en la que cada estudiante
recibe 4 horas presenciales de clase a la semana distribuídas del siguiente modo:
\begin{itemize}
  \item 2 horas dedicadas al estudio de contenidos teóricos
  \item 1 hora dedicada a la resolución de problemas
  \item 1 hora de prácticas
\end{itemize}
Se describe a continuación el tipo de actividades que se desarrolla en cada una de este tipo de sesiones.

Las sesiones destinadas a contenidos teóricos se imparten con el formato de clase expositiva y se dedican, 
con el uso de transparencias que el alumnado tiene disponibles a través del aula virtual, al estudio de los 
contenidos de la asignatura. 
Para cumplir con el requisito del número de créditos que cada estudiante ha de cursar en inglés en la 
titulación, la mayor parte del material que se utiliza en la asignatura está en ese idioma y se incentiva 
también el uso del mismo en todas las actividades que se desarrollan, particularmente en todo lo relacionado 
con la materia de programación.
Junto a las transparencias, el alumnado dispone de una serie de pequeños programas, disponibles a través de un
repositorio público, que sirven de ilustración a los conceptos que se estudian en las sesiones teóricas. 
Se recomienda al alumnado estudiar esos programas para afianzar los contenidos expuestos.

En las sesiones de problemas el profesorado utiliza un terminal cuya pantalla se proyecta a toda la clase para
solucionar de forma práctica algunos problemas seleccionados, resolviendo al mismo tiempo las dudas que puedan 
surgir en esa resolución. 

La Guía docente establece que por cada hora de trabajo presencial, cada estudiante debería dedicar en 
promedio 1,5 horas de trabajo autónomo, de modo que se espera unas 6 horas de trabajo autónomo semanal por 
parte de cada estudiante. 
A pesar de que el peso de los contenidos prácticos de la asignatura suponen solo un 20\% del total de la
calificación de la asignatura, se espera que la mayor parte de ese tiempo se destine al diseño y desarrollo 
de programas de complejidad creciente conforme avanza el desarrollo del curso, y que se evalúan en las
sesiones prácticas.

Cada semana al alumnado se le propone la realización de una práctica, consistente en un cierto número (cinco
es un número habitual) de programas en C++ relacionados con algún tema estudiado.
Esos programas tienen el propósito de servir de ``entrenamiento'' para que el alumnado afiance conocimientos a
lo largo de la semana de la que disponen para realizar esos ejercicios.
La sesión semanal de prácticas se destina a la evaluación de esos conocimientos a través de la realización en
el laboratorio de prácticas de ejercicios de programación de complejidad similar a los que han sido propuestos 
con antelación.
En las últimas prácticas de la asignatura los ejercicios de evaluación suelen ser modificaciones o
ampliaciones de los que se han propuesto para realizar con antelación.

% 10 minutos para cuestionarios
%%%%%%%%%%%%%%%%%%%%%%%%%%%%%%% Section %%%%%%%%%%%%%%%%%%%%%%%%%%%%%%% 
\section{Motivación}
Más allá del ineludible aprendizaje de los conceptos básicos de la materia, 
es un hecho bien conocido que la práctica es fundamental para aprender a programar ordenadores. 
Habitualmente el profesorado asigna problemas de programación al alumnado para ayudarles a adquirir esta 
destreza, y el incremento de las habilidades de una programadora pasa ineludiblemente por muchas horas de dedicación 
a la realización de programas de complejidad creciente.
Forzando un poco las similitudes podríamos decir que la programación se asemeja a las habilidades para
conducir un automóvil: cualquier persona con permiso para conducir puede decir que conduce correctamente, pero
acreditar que se es una buena conductora requiere muchas horas de práctica y exposición a situaciones
infrecuentes.
Del mismo modo cualquier informático dirá que sabe programar pero sus habilidades en esta materia dependerán
muy directamente de las horas de práctica que haya dedicado a la misma.
Siguiendo esta idea, se recomienda al alumnado de ASIG que realice cuantos ejercicios prácticos
sean capaces para, de forma progresiva, ir incrementando sus capacidades como programadoras.

Las prácticas de la asignatura se convierten pues en una oportunidad para que el alumnado mejore destrezas y
habilidades que le capaciten para abordar los contenidos de asignaturas de cursos posteriores de la
titulación.

La masificación de los grupos de laboratorio de prácticas, con grupos de hasta 20 estudiantes por sesión es
la mayor dificultad para la evaluación de esos trabajos prácticos.
Esta situación es compartida por muchas otras asignaturas de la titulación que tienen una significativa
componente práctica en sus contenidos, de modo que algunas cuestiones que en este trabajo se plantean
son comunes a muchas asignaturas e incluso a otras titulaciones.

Una vez realizadas tres prácticas iniciales en las que el alumnado se familiariza con el sistema operativo
Linux, con el entorno de máquina virtual en el que desarrollará sus programas y con el editor \texttt{vim}, 
que es el que se utiliza inicialmente, todas las prácticas restantes abordan contenidos de programación que
cubren los siguientes tópicos:
\begin{itemize}
  \item Primeros programas y conceptos básicos
  \item Expresiones y tipos de datos
  \item Alternativas
  \item Iteraciones
  \item Funciones
  \item Cadenas de texto (\texttt{std:string})
  \item \texttt{std::array} y \texttt{std::vector}
  \item Ficheros
  \item Introducción a la Programación Orientada a Objetos
\end{itemize}

Desde hace ya varios cursos en ASIG se viene usando la plataforma \jutge{} (\textit{juez}) 
\cite{Petit:Jutge:2018} para la evaluación de las prácticas.
\jutge{} ha sido desarrollado en la Universidad Politécnica de Catalunya, orientado tanto al profesorado 
como al alumnado.
La plataforma aloja una gran cantidad de problemas (aproximadamente unos 2100) que cubren diversos
tópicos incluyendo entre ellos los fundamentos de la programación.
Los problemas están perfectamente descritos y contienen un conjunto de tests que el código de usuario ha de
verificar.
Como una de sus características, \jutge{} hace hincapié en el trabajo del alumnado, por lo que 
resulta especialmente útil para reforzar el enfoque de aprender haciendo, en nuestro caso, programando.

El modo de funcionamiento de \jutge{} requiere que cuando un estudiante resuelve un problema suba el código
fuente de su solución a la plataforma
\footnote{\textit{Jutge.org} Home Page \href{https://jutge.org/}{\scriptsize{\texttt{https://jutge.org/}}}}
donde se comprueba que sea correcta pasando los correspondientes tests, de los cuales algunos son públicos
y otros privados.
A una solución se le asigna el veredicto \textit{AC} (\textit{Accepted}) cuando pasa todos los tests existentes para el problema.
En su cuenta de \jutge{} cada estudiante dispone de un cuadro de mandos en el que se recopila el número de
envíos que ha realizado, el de problemas aceptados y rechazados así como diversos gráficos que muestran la
evolución en su trabajo con los problemas de la plataforma.

Este modo de trabajo se aprovecha en las prácticas de ASIG: en cada sesión de evaluación se le pide al alumnado
que resuevla un pequeño número de problemas (programas) de \jutge{}.
La evaluación de la sesión depende no solo del número de problemas resuelto (no suelen ser más de dos o tres)
sino de la calidad del código desarrollado.
\jutge{} comprueba exclusivamente (mediante la comprobación de los correspondientes tests) que el programa evaluado 
funcione correctamente, mientras que en ASIG se enfatizan otros aspectos del código que nos parecen tanto o más 
relevantes que el propio funcionamiento del mismo, que obviamente es una condición ineludible.
Algunos de los requisitos que se exigen a los programas de prácticas de ASIG se definen en la Guía de estilo de 
referencia que se utiliza en la asignatura 
%\cite{Google::GSG}
\footnote{Google C++ Style Guide\\ \href{https://google.github.io/styleguide/cppguide.html}{\scriptsize{\texttt{https://google.github.io/styleguide/cppguide.html}}}}.

Se relacionan a continuación algunos de los requisitos exigidos:
\begin{itemize}
\item Correcto sangrado del código
\item Adecuado uso de espacios y signos de puntuación en el código
\item Adhesión a las reglas de nombrado de identificadores de variables, funciones, clases, etc.
\item Todos los identificadores utilizados (salvo excepciones puntuales) han de ser significativos, evitando el
      uso de ``identificadores de un único carácter''.
\item Los programas deben escribirse de forma modular incorporando diversas funciones en la solución.
\item Todos los ficheros, funciones y métodos del código han de incluir un breve prólogo con comentarios en
      formato \textit{Doxygen} exponiendo la información más relevante sobre el elemento (función, clase, fichero, ...) en
      cuestión.
\item La compilación de todos los programas ha de automatizarse mediante el uso de herramientas como \texttt{make} o \texttt{cmake}.
\item Se enfatiza que los parámetros de tipos estructurados que se pasen a una función/método se pasen como referencias,
      cualificadas como constantes, si procede.
\item Se promueve que los métodos definidos en las clases sean \textit{const friendly}.
\end{itemize}
Estos requisitos se introducen al alumnado de forma progresiva y son parte fundamental de la evaluación de cada programa, 
una vez satisfecho el requisito de un correcto funcionamiento.

Un problema recurrente en todas las asignaturas que requieren la evaluación de prácticas de programación es la
infracción por parte de algunos estudiantes de las reglas de código de conducta que establecen que los
trabajos presentados a evaluación han de ser programas originales realizados por sus autores.
Ante la dificultad de acreditar fehacientemente esta condición en una sesión de evaluación con un elevado
número de estudiantes en el aula de prácticas y con un tiempo tan limitado se ha optado por valorar
casi exclusivamente el trabajo que el estudiante realiza \textit{in situ} en la sesión de evaluación, y no los
ejercicios que durante la semana ha realizado en calidad de preparación para esa evaluación.
Se permite al alumnado en la sesión de evaluación el uso de recursos disponibles en internet (documentación,
foros, etc.), puesto que entendemos que el alumnado ha de ser evaluado en condiciones de trabajo similares a 
las que tendrá en un entorno profesional.

Es en este punto donde queremos pasar a considerar los asistentes de programación basados en Inteligencia Artificial (IA).
En junio de 2021, GitHub lanzó \textit{Copilot} 
%\cite{Friedman:2021:IGC}
%\cite{Dakhel:2022:GCA},
\cite{Nguyen:2022:AnEE},
un ``programador de pares de IA'' capaz de generar
código en diversos lenguajes a partir de cierto contexto como comentarios, nombres de funciones y código adyacente. 
\textit{Copilot} se basa en un modelo que se entrena con código abierto \cite{Chen:2021:ELL}.

En febrero de 2022 \textit{DeepMind} publicó \textit{Alphacode}, otro sistema basado en IA
generativa que puede competir con un humano en la resolución de problemas sencillos de programación.
Según resultados publicados en \textit{Sience} \cite{Li:2022:CCG}, \textit{Alphacode} gana en un 50\%
de ocasiones a humanos en competiciones de resolución de problemas de programación.

El pasado 30 de noviembre, \textit{OpenAI} lanzó \ChatGPT{}
\cite{Zhang:2020:chatgpt}, 
un \textit{chatbot} interactivo de propósito general basado en GPT-3.5
\cite{Floridi:2020:GPT-3},
un modelo lingüístico autorregresivo de tercera generación que utiliza \textit{DeepLearning} para producir textos 
similares a los escritos por humanos.

Tanto \ChatGPT{} como \textit{AlphaCode} son ``grandes modelos lingüísticos'', es decir, sistemas basados en 
redes neuronales que aprenden a realizar una tarea a partir de ingentes cantidades de texto generado por humanos. 
De hecho, ambos sistemas utilizan prácticamente la misma arquitectura, siendo la principal diferencia entre
ellos el conjunto de datos con que son entrenados, lo cual los dirige a diferente tipo de tareas.
Posiblemente debido a que se trata de un asistente de propósito general, capaz de discutir sobre materias tan dispares 
como derecho, filosofía o programación de ordenadores, ha sido el lanzamiento de \ChatGPT{} el que ha suscitado un mayor
interés, no solo en la comunidad informática sino también en medios de comunicación de caracter generalista
\cite{Perez:2022:FBM} 
atrayendo la atención de más de un millón de usuarios apenas cinco días después de su lanzamiento.

A pesar de que hay un consenso generalizado en la comunidad informática respecto a que, al menos en su estado
actual, estos asistentes no van a suponer la desaparición de los programadores de ordenadores
\cite{Castelvecchi:2022:ACaA}, 
también es incuestionable el papel que estos asistentes pueden llegar a tener en la toma de decisiones en
diversos campos 
\cite{Kung:2022:PCU}.

%con patrones de codificación inseguros, lo que da lugar a la posibilidad de sintetizar código que contenga estos patrones indeseables".

En la siguiente sección se presentan algunas experiencias significativas realizadas con \ChatGPT{} en el ámbito
de los ejercicios de programación de las prácticas de ASIG.
%%%%%%%%%%%%%%%%%%%%%%%%%%%%%%% Section %%%%%%%%%%%%%%%%%%%%%%%%%%%%%%% 
\section{Experiencias con \ChatGPT{}}
Aunque el estudio que se ha realizado no es exhaustivo, se ha utilizado \ChatGPT{} para tratar de resolver
diversos problemas de \jutge{}
\footnote{
Todos los programas que se mencionan en este trabajo se han alojado convenientemente documentados en un repositorio privado.
Cualquier persona interesada en acceder a esos programas puede solicitarlo por correo electrónico al autor 
(\texttt{anonimo@anonimo.es})
%(\texttt{fsande@ull.es})
}. 
Debido a limitaciones de espacio no podemos reseñar todos los experimentos realizados, de modo que 
se exponen solamente los que consideramos más relevantes o de interés para comprender el modo de funcionamiento 
del bot y sus capacidades para resolver ejercicios concretos.

Todos los enunciados de problemas de \jutge{} están públicamente disponibles en la
plataforma 
\footnote{Jutge Problems \href{https://jutge.org/problems/}{\scriptsize{\texttt{https://jutge.org/problems/}}}}
y cada uno de ellos cuenta con una especificación clara y precisa así como de un conjunto de tests públicos.
Se han descrito los problemas a \ChatGPT{} tanto en inglés (idioma en que están descritos la mayoría de
los problemas de \jutge{}) como en español.
Para ello se ha iniciado una nueva sesión con el bot en uno u otro idioma dependiendo del caso.
Si bien las soluciones que entrega el sistema no son exactamente las mismas, no se han hallado diferencias
significativas en cuanto a la calidad tanto algorítmica como de estilo en las soluciones obtenidas.
%%%%%%%%%%%%%%%%% Code %%%%%%%%%%%%%%%%%%%%%%%%%%%%%%%%%%%%%
\lstinputlisting[float=*t, caption={Test de primalidad suministrado por \ChatGPT{}. Complejidad $O(\sqrt(n))$}, label=code:prime]{prime.cc}
%%%%%%%%%%%%%%%%%%%%%%%%%%%%%%%%%%%%%%%%%%%%%%%%%%%%%%%%%%%%
Para que el bot resuelva cualquiera de los problemas no ha resultado necesario suministrarle información
adicional a la que figura en el propio enunciado de \jutge{}, habiéndosele entregado al bot la información
correspondiente a la salida que el programa debiera entregar para los diferentes tests públicos.

Por otra parte, si no se le especifica explícitamente, las soluciones que produce \ChatGPT{} generalmente no
siguen reglas de estilo o documentación específicas, pero si en el enunciado del problema se incluyen
estas reglas como requisito, la solución que entrega el bot se ajusta a las mismas.
Es más, al tratarse de un bot interactivo, si la solución que entrega el sistema no es satisfactoria en algún
sentido, se le puede pedir que refactorice el código para cumplir con algún requisito y en ese caso en general
el sistema adecúa su respuesta para cumplir con la solicitud.

%Así por ejemplo, en el problema
%\href{https://jutge.org/problems/P36430_en/statement}{\textit{Fermat's last theorem (1)}}
%se especifica ``\textit{Input consists of four natural numbers a, b, c, d such that $a \leq b$ and $c \leq d$.}'' 
%lo cual produce que la primera versión que el bot entrega para la solución de ese problema utiliza 
%\textit{a, b, c,} y \textit{d} para los identificadores de esas variables.

Consideremos en primer lugar el problema
\href{https://jutge.org/problems/P36430_en/statement}{\textit{Fermat's last theorem (1)}}
para el cual se muestra a continuación la descripción que se entregó inicialmente
a \ChatGPT{}, copiada casi literalmente de \jutge{}:
%%%%%%%%%%%%%%%%%%%%%%%% Quote %%%%%%%%%%%%%%%%%%%%%%%% 
\begin{quote}
   \small{
     \textit{
   A famous theorem of the mathematician Pierre de Fermat, proved after more than 300 years, 
states that, for any natural number $n \geq 3$, there is no natural solution (except for x=0 or y=0) to the equation
 $x^n + y^n = z^n$

For $n=2$, by contrast, there are infinite non-trivial solutions. 
For instance, $3^2 + 4^2 = 5^2$, $5^2 + 12^2 = 13^2$, $6^2 + 8^2 = 10^2$, ...

Write a C++ program that, given four natural numbers a, b, c, d with $a \leq b$ and $c \leq d$, prints a natural solution to the equation
  $x^2 + y^2 = z^2$
such that $a \leq x \leq b$ and $c \leq y \leq d$.

The program input consists of four natural numbers a, b, c, d such that $a<=b$ and $c<=d$.

The program output should be a line with a natural solution to the equation
  $x^2 + y^2 = z^2$
that fulfills $a \leq x \leq b$ and $c \leq y \leq d$. 

If there is more than one solution, print the one with the smallest x. 

If there is a tie in x, print the solution with the smallest y. 

If there are no solutions, print ``No solution!''.

For example, if the input is:
\texttt{2 5 4 13}

The output should be

\texttt{3} \texttt{\^} \texttt{2 + 4} \texttt{\^} \texttt{2 = 5} \texttt{\^} \texttt{2}

and if the input is:
\texttt{1 1 1 1}

     The output should be 
\texttt{No solution!}

     }
   }
\end{quote}
%%%%%%%%%%%%%%%%%%%%%%%%%%%%%%%%%%%%%%%%%%%%%%%%%%%%%%% 
%%%%%%%%%%%%%%%%%%%%%%%% Cita %%%%%%%%%%%%%%%%%%%%%%%% 
%\begin{center}
%  \begin{minipage}{\linewidth}
%    {\small
%    A famous theorem of the mathematician Pierre de Fermat, proved after more than 300 years, 
states that, for any natural number $n \geq 3$, there is no natural solution (except for x=0 or y=0) to the equation
 $x^n + y^n = z^n$

For $n=2$, by contrast, there are infinite non-trivial solutions. 
For instance, $3^2 + 4^2 = 5^2$, $5^2 + 12^2 = 13^2$, $6^2 + 8^2 = 10^2$, ...

Write a C++ program that, given four natural numbers a, b, c, d with $a \leq b$ and $c \leq d$, prints a natural solution to the equation
  $x^2 + y^2 = z^2$
such that $a \leq x \leq b$ and $c \leq y \leq d$.

The program input consists of four natural numbers a, b, c, d such that $a<=b$ and $c<=d$.

The program output should be a line with a natural solution to the equation
  $x^2 + y^2 = z^2$
that fulfills $a \leq x \leq b$ and $c \leq y \leq d$. 

If there is more than one solution, print the one with the smallest x. 

If there is a tie in x, print the solution with the smallest y. 

If there are no solutions, print ``No solution!''.

For example, if the input is:
\texttt{2 5 4 13}

The output should be

\texttt{3} \texttt{\^} \texttt{2 + 4} \texttt{\^} \texttt{2 = 5} \texttt{\^} \texttt{2}

and if the input is:
\texttt{1 1 1 1}

     The output should be 
\texttt{No solution!}

%    }
%  \end{minipage}
%\end{center}
%%%%%%%%%%%%%%%%%%%%%%%%%%%%%%%%%%%%%%%%%%%%%%%%%%%%%% 
Con esta especificación, el bot entrega una solución que es aceptada (\textit{AC}) por \jutge{} pero que no cumple algunos de los
requisitos que se exigen en ASIG. 
En particular, el código de la solución utiliza \textit{a, b, c,} y \textit{d} como identificadores de
variables, tal como frecuentemente hacen los estudiantes, puesto que así figuran en la especificación.
No obstante, después de sucesivas interacciones con el bot en el que se le indica:
\begin{enumerate}
  \item Could you please avoid the use of single character identifiers and use meaningful names instead?
  \item Please, use a function called from main() in your solution
  \item Can yoy make your code compliant with the Google Style Guide for C++?
  \item Can you avoid the use of ``\texttt{using namespace std}''?
  \item Can you write a space on both sides of any binary operator, as the Google Style Guide for C++ requires?
  \item Can you finally include Doxygen format header comments in the code?
\end{enumerate}
Se obtiene un código que es aceptado por \jutge{} y que cumple los requisitos exigidos al alumnado de ASIG en la
evaluación de sus prácticas.

Consideremos a continuación el problema 
\href{https://jutge.org/problems/P48713_en}{\textit{Primality}}.
Se trata de determinar si cada uno de los números naturales de una secuencia es o no primo.
Al entregar a \ChatGPT{} el enunciado del problema en inglés junto con los tests del mismo, tal como figuran en \jutge{}, 
la solución que se obtuvo fue la de fuerza bruta consistente en probar para el número $N$ todos los divisores en el 
rango $[2, N-1]$.
Esta solución no recibe el veredicto ``\textit{Accepted}'' en \jutge{} sino que la plataforme indica
como veredicto \textit{Execution Error (time limit exceeded)}.
Ello se debe a que espera un algoritmo más eficiente para este cómputo.

Esta solución es la que cabría esperar de un estudiante de primer curso de informática o de un programador
inexperto.
Con frecuencia encontramos estudiantes que, para este problema ofrecen un algoritmo óptimo y ello es una pista 
para detectar que, posiblemente han hallado la solución en algún foro. 
Al preguntar la razón por la que no recorre todo el rango de búsqueda cabría esperar una
respuesta en la que el programador indique que ha investigado el problema y aprendido sobre el mismo, pero es
frecuente una respuesta del tipo ``\textit{lo he probado y he observado que funcionaba}'', que obviamente no se
considera adecuada en la evaluación de un ejercicio práctico.

En el propio enunciado del problema se indica una pista para un algoritmo de menor complejidad.
Si a \ChatGPT{} se le indica:
\textit{Could you optimize the is\_prime() function for a better performance?}
el bot modifica la función, entregando en este caso una versión que sí es aceptada en \textit{Jutge.org}.
A pesar de haberlo indicado en el enunciado del problema, la función entregada no cumple con el convenio que
establece la Guía de Estilo de Google, pero si se le indica esa circunstancia, el bot corrige el identificador
de la función. 
El código completo de la función es el que se muestra en el Listado \ref{code:prime}.

%%%%%%%%%%%%%%%%% Code %%%%%%%%%%%%%%%%%%%%%%%%%%%%%%%%%%%%%
\lstinputlisting[caption={Declaración preliminar de la clase \textit{Box}}, label=code:boxh]{box1.h}
%%%%%%%%%%%%%%%%%%%%%%%%%%%%%%%%%%%%%%%%%%%%%%%%%%%%%%%%%%%%
En el caso de \textit{Primality}, en una sesión diferente con \ChatGPT{} se le entregó el enunciado en español y
en ese caso, el bot devolvió directamente la solución con la optimización del código, pero la solución no cumplía
con el estándar de Estilo en cuanto a la colocación de las llaves de apertura y cierre de bloques en C++.

Particularmente interesante nos ha resultado el caso del experimento realizado con el problema
\href{https://jutge.org/problems/P73501_en/statement}{\textit{Increasing Pairs}}, 
consistente en calcular el número de pares de números consecutivos en una secuencia en los que el segundo
número del par sea mayor que el primero.
Suministrándole a \ChatGPT{} la breve descripción del problema en inglés tal cual figura en \jutge{}, el bot
entrega una solución que, si bien pasa los tests públicos no consigue el veredicto \textit{AC} del juez porque
falla en algún test privado.
Hay que tener en cuenta que solo se sabe que el programa falla para algún test (secuencia de números) pero no
se sabe más sobre ella.
Informando al bot de esta circunstancia, éste responde con una segunda versión de la solución en la que ha
resuelto algún caso particular en el que, efectivamente admite que su primera versión fallaría.
Esta segunda versión adolece del mismo problema que la anterior: falla en algún test privado de \jutge{}, que
por lo tanto no la acepta.
Se le vuelve a indicar el fallo a \ChatGPT{} y finalmente produce una solución que \jutge{} valida.

%%%%%%%%%%%%%%%%% Code %%%%%%%%%%%%%%%%%%%%%%%%%%%%%%%%%%%%%
\lstinputlisting[float=*t, caption={Contenido del fichero \texttt{box.cc} generado por \ChatGPT{}}, label=code:boxc]{box1.cc}
%%%%%%%%%%%%%%%%%%%%%%%%%%%%%%%%%%%%%%%%%%%%%%%%%%%%%%%%%%%%
Este modo de trabajo, interactuando con el bot para refinar la solución en sucesivas iteraciones fue el que se
utilizó para el experimento que se realizó con un problema consistente en la implementación en C++ de una
clase para representar ``cajas''. 
Se trata de un problema de programación que ha figurado como ejercicio de evaluación, en este caso
en exámenes escritos, del alumnado de ASIG.
Se exponen a continuación en español las sucesivas peticiones que se realizaron a \ChatGPT{} en inglés:
\begin{enumerate}
  \item A continuación (véase el Listado \ref{code:boxh}) se muestra la definición parcial en C++ de una clase \textit{Box}
    que representa una caja.
  \item Queremos que el código siga las recomendaciones de la Guía de Estilo de Google para C++.
        Completa la definición de la clase \textit{Box} incluyendo los constructores y métodos que consideres
        adecuado.

        El bot responde incluyendo en la definición de la clase un par de constructores, métodos mutadores 
        (\textit{setters}) y un par de métodos que permiten calcular volumen y área de la superficie de la caja.
        \ChatGPT{} informa asimismo de que declaración e implementación de la clase deberían ubicarse en sendos ficheros
        con extensiones \texttt{.cpp} y \texttt{.h}. 
  \item Escribe ahora la definición de los métodos que deberían incluirse en el fichero \texttt{.cc}.
  \item Implementa la sobrecarga del operador de inserción (\texttt{$<<$}).
  \item ¿Podrías añadir comentarios en formato \textit{Doxygen} para la función \texttt{operator$<<$}?
  \item Implementa la sobrecarga del operador de extracción (\texttt{$>>$}).
  \item Implementa la sobrecarga de operadores requerida para comprobar si una caja está antes que otra 
        en un listado ordenado por volumen de mayor a menor.
  \item Escribe el código de una función \textit{main()} en la que se instancien 2 objetos de tipo \textit{Box} 
        con volúmenes diferentes y se muestre por pantalla cuál de los dos se encuentra antes en el listado de volumen.
  \item Escribe un fichero \textit{Makefile} para compilar el programa usando tres ficheros:
    \texttt{main-box.cc}, \texttt{box.cc} y \texttt{box.h}.
        El compilador que queremos usar es \textit{g++}.
  \item ¿Qué ficheros deberían incluirse en \texttt{main.box.cc} para que el programa compile correctamente?.
  \item ¿Qué declaraciones debería incluir en el fichero \texttt{box.h} para que el programa compile
    correctamente?.
  \item Al compilar obtengo el siguiente error:

        \texttt{
In file included from main-box.cc:1:
box.h:94:6: error: ‘ostream’ in namespace ‘std’ does not name a type
        }

        \ChatGPT{} responde indicando:

        \textit{
        The error message you provided indicates that the compiler is unable to find the definition of std::ostream. 
        This is likely because the $<$iostream$>$ header file, which defines the std::ostream class, has not been included in the box.h file.
        To fix this error, you need to include the <iostream> header file at the beginning of box.h. 
        }
\end{enumerate}
%
%     Para sobrecargar el operador de extracción ¿es perferible hacerlo como una función friend o usando una función ordinaria?
%     To overload the extraction operator, is it preferable to do it as a friend function or using an ordinary function?
%     Respuesta:
%     It is generally considered best practice to overload the extraction operator (operator>>) as a non-member function, 
%     rather than a friend function. 
%     This is because the extraction operator is typically used to extract data from an input stream, and making it a friend 
%     function of a class would allow it to access the class's private and protected data members, which could be a security risk. 
%     Additionally, making it a non-member function allows for greater flexibility and encapsulation, as it can be used with any 
%     object that can be passed to it as an input stream, rather than being limited to a specific class.
A cada una de las peticiones anteriores el bot responde entregando sucesivos fragmentos de código con sus correspondientes 
explicaciones.
Una vez corregido el último error señalado, el programa completo con los tres ficheros puede compilarse con
\texttt{make} obteniéndose un código ejecutable que funciona de acuerdo a la especificación.
El listado \ref{code:boxc} muestra el contenido del fichero \texttt{box.cc} en el que se han eliminado los comentarios
en formato \textit{Doxygen} que el bot incluyó antes de cada una de las funciones.

%%%%%%%%%%%%%%%%%%%%%%%%%%%%%%% Section %%%%%%%%%%%%%%%%%%%%%%%%%%%%%%% 
\section{Conclusiones}
Como muestran los experimentos reflejados en la sección anterior, la capacidad de \ChatGPT{} para resolver problemas sencillos
de programación como los que se utilizan en los primeros cursos del Grado en Ingeniería Informática es
sobresaliente.
Podemos afirmar que el asistente es capaz de resolver correctamente la mayoría de los problemas especificados en
la plataforma \jutge{}, cumpliendo no solo con el requisito de un correcto funcionamiento sino también con los
de buenas prácticas y estilo de programación que se exigen en ASIG.

Queda patente que en manos de un programador experimentado, el sistema es -ya en su estado actual- una
herramienta muy valiosa para la realización de tareas que pueden entrañar un cierto grado de sofisticación.
Interactuando con el sistema es posible refinar en pasos sucesivos una solución dirigiéndola adecuadamente
hasta lograr satisfacer unos requisitos predeterminados por el usuario.

Una relación no exhaustiva de preguntas que los experimentos expuestos anteriormente motivan y sobre las cuales 
debemos reflexionar son las siguientes:
\begin{itemize} 
\item ¿Suponen los avances en IA conversacional el final de la necesidad de la programación de ordenadores como hasta ahora
  la hemos conocido?
\item ¿Debemos ignorar la existencia de los asistentes basados en IA o se necesita por el contrario, incorporarlos a
    nuestra práctica docente?
\item ¿Qué cambios debemos introducir en las prácticas docentes de la enseñanza de la programación?
\item ¿En qué momento del itinerario formativo en materia de programación debiera incorporarse el conocimiento
  y manejo de los asistentes de programación?
\item ¿Son útiles los asistentes en los niveles iniciales del aprendizaje de la programación o por el
  contrario su uso debe aplazarse a niveles posteriores?
\item ¿Hay algún peligro en el uso de asistentes que debiéramos tener en cuenta a la hora de exponer al
  alumnado a estas herramientas?
\end {itemize} 

No es objetivo de este trabajo ofrecer respuestas a estas y otras preguntas que cabe plantear, sino propiciar
la discusión y el intercambio de ideas sobre el nuevo escenario que tenemos que afrontar en las aulas de forma
inmediata.

Si tal como muchos expertos vaticinan 
\cite{Welsh:2023:TEoP}, 
el desarrollo de grandes modelos de lenguajes supondrá en breve la desaparición de los programadores de
ordenadores, la enseñanza de la programación habrá de modificarse para adaptarse a esta nueva realidad. 
Del mismo modo que en la actualidad un desarrollador de software en general desconoce los detalles del
hardware en el que se ejecutan sus programas, es posible que en el futuro un ingeniero de software no
tenga que conocer los detalles del código que interviene en un sistema complejo.
Habrá en este caso que desplazar progresivamente el énfasis en los contenidos de programación hacia 
conocimientos relacionados con ingeniería y arquitectura del software.

Hurtar al alumnado el uso de los asistentes basados en IA generativa en sus prácticas de programación
\cite{Cassidy:2023:AUR}, 
además de ser un esfuerzo vano, creemos que sería un enfoque profundamente erróneo.
Se trata de herramientas que ya forman parte del bagaje del que un profesional dispone para desarrollar su trabajo.
Del mismo modo que carece a nuestro juicio de sentido evaluar las capacidades de programación de un estudiante
mediante exámenes tradicionales con bolígrafo y papel, tampoco tiene sentido prohibirles el uso de
asistentes que en el futuro tendrán que conocer y utilizar.
En lugar de prohibir a los estudiantes el uso de estos sistemas, creemos que lo que ha de hacerse es ayudar
al profesorado y al alumnado a incorporar estas nuevas herramientas como elementos de apoyo al aprendizaje.

La situación que ahora enfrentamos no es nueva: la aparición de Wikipedia 
%(nos referimos en este caso a ámbitos diferentes del de la programación de ordenadores) 
o foros de discusión como
\textit{StackOverflow} en los que se ofrecen soluciones a multitud de problemas de diferente tipo supuso en su
momento una ampliación del conjunto de herramientas disponibles a la hora de programar y de aprender a
programar.
Es lo mismo que ocurre hoy en día, pero agravado si cabe, porque las capacidades que en la
actualidad muestran los nuevos asistentes son realmente impresionantes y es difícil aventurar lo que pueden
llegar a ofrecer en un corto o medio plazo.
De hecho, son las prestaciones que ya ofrecen estos asistentes lo que más ha llamado la atención a
especialistas en IA: se esperaba que estos adelantos se fueran a producir pero ha causado cierta sorpresa el
haberlos logrado en la actualidad.

Si bien la llegada de los asistentes basados en IA a las aulas no implicará que el profesorado deje de ser
necesario, sí es cierto que estas tecnologías van a impactar en la praxis docente y hemos de adaptar nuestras
metodologías para incorporar estos cambios.
Integrar estos asistentes en las metodologías docentes e instruir a profesorado y alumnado acerca de su 
funcionamiento y de sus posibles usos será clave para lograr una utilización correcta de estos sistemas.
\balance{}
\bibliographystyle{jenui}
%\nocite{*}
\bibliography{Jenui2023}
\end{document}
