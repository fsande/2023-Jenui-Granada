\documentclass[twocolumn,twoside,a4paper, 10pt]{article}
%Revisado Noviembre 2018

\usepackage[utf8]{inputenc}
\usepackage[T1]{fontenc}
\usepackage[spanish]{babel}
\usepackage{balance}
\usepackage{jenuia4}
\usepackage{url}
\usepackage{graphicx}

\usepackage{hyperref}

\title{El impacto de la IA en la enseñanza de la programación}

%\author{\normalsize 
%\begin{tabular}{@{\extracolsep{3mm}}cc}
%{\large Joe Miró Julià }                  & {\large Mercedes Marqués}\\
%Departament de Matemàtiques i Informàtica & Departamento de Ing. y Ccia. de los Comput.\\
%Universitat de les Illes Balears          & Universitat Jaume I\\
%07122 Palma de Mallorca                   & Castellón\\
%\url{joe.miro@uib.es}                     & \url{merche.marques@uji.es}
%\end{tabular}
%}

\author{ \small
\begin{tabular}{@{\extracolsep{3mm}}c}
\large Francisco de Sande \\
Departamento de Ingeniería Informática y de Sistemas \\
Universidad de La Laguna \\
38200 La Laguna. S/C de Tenerife \\
fsande@ull.es
\end{tabular}
}

\date{}

%%%  Referencias
% https://aihub.csic.es/inteligencia-artificial-en-educacion-golem-creativo-o-destructor/

\begin{document}
\maketitle
\thispagestyle{empty}

\begin{abstract}
\noindent Abstract
\end{abstract}

\section*{Abstract}
\noindent Abstract Pedirle que los ids del código sean significativos.

\section*{Palabras clave}
\noindent Programación IA Enseñanza Informática Evaluación

\section{Pruebas}
Jutge \cite{URL::Jutge, Petit:Jutge:2018}

Comenzar a hablar de lo que hacems en IB para evaluar las prácticas de programación.
Descripción de la asignatura, temario, etc.

chatGPT \cite{Zhang:2020:chatgpt, Castelvecchi:2022:ACaA}

AlphaCode \cite{Li:2022:CCG}


En junio de 2021, GitHub lanzó Copilot \cite{Friedman:2021:IGC}, un "programador de pares de IA" que genera 
código en diversos lenguajes a partir de cierto contexto como comentarios, nombres de funciones y código adyacente. 
Copilot se basa en un modelo que se entrena con código abierto \cite{Chen:2021:ELL}, incluido "código público... 
con patrones de codificación inseguros, lo que da lugar a la posibilidad de sintetizar código que contenga 
estos patrones indeseables".

\section{Motivación}
\textit{Informática Básica} (IB, de ahora en adelante) es una asignatura de 6 créditos que se imparte en el primer cuatrimestre 
del primer curso del Grado en Ingeniería Informática en la Escuela Superior de Ingeniería y Tecnología de la
Universidad de La Laguna.

Se trata de la primera asignatura (y única en ese primer cuatrimestre) de perfil eminentemente informático que
cursa el alumnado del título de Grado.
El número de estudiantes que habitualmente cursa la asignatura está en torno a los 250. 
Los contenidos de la asignatura pueden consultarse en la Guía Docente \cite{ULL:2022:GD} de la misma.
Al margen de un par de temas dedicados a una introducción a las Bases de Datos, Redes y Sistemas Operativos,
el grueso de los contenidos (en torno a 12 de las 15 semanas del cuatrimestre) se dedican a introducir al
alumnado en la materia de Programación, siendo C++ el lenguaje vehicular elegido para estudiar la materia.
Se trata de una asignatura con una importante proporción de contenidos prácticos en la que cada estudiante
recibe 4 horas presenciales de clase a la semana distribuídas del siguiente modo:
\begin{itemize}
  \item 2 horas dedicadas al estudio de contenidos teóricos
  \item 1 hora dedicada a la resolución de Problemas
  \item 1 horas de Prácticas
\end{itemize}
La Guía docente establece que por cada hora de trabajo presencial, cada estudiante debería dedicar en 
promedio 1,5 horas de trabajo autónomo, de modo que se espera unas 6 horas de trabajo autónomo semanal por 
parte de cada estudiante. 
Se describe a continuación el tipo de actividades que se desarrolla en cada una de este tipo de sesiones.

Las sesiones destinadas a contenidos teóricos se imparten con el formato de clase expositiva y se dedican, 
con el uso de transparencias que el alumnado tiene disponibles a través del aula virtual, al estudio de los 
contenidos de la asignatura. 
Junto a las transparencias, el alumnado dispone de una serie de pequeños programas que sirven de ilustración a
los contenidos que se estudian en clase. 
Se recomienda al alumnado estudiar esos programas para afianzar sus conocimientos de cada tema.

En las sesiones de problemas el profesorado utiliza un terminal cuya pantalla se proyecta a toda la clase para
resolver algunos problemas seleccionados, resolviendo al mismo tiempo las dudas que puedan surgir en esa
resolución. 

Cada semana al alumnado se le propone la realización de una práctica, consistente en un cierto número (del
orden de cinco) de programas en C++ relacionados con el tema que se está estudiando.
Esos programas tienen el propósito de servir de "entrenamiento" para el alumnado para afianzar conocimientos
que se evalúan en directo en la sesión semanal (1 hora) de evaluación de prácticas.

El alumnado tiene una semana para desarrollar sus programas que son evaluados en la sesión de evaluación de prácticas (2 horas). 

En esa evaluación se tiene en cuenta no solo el trabajo que el alumnado ha desarrollado con antelación sino también se les pide que modifiquen o hagan cambios en el programa que presentan a evaluación para confirmar que conocen perfectamente el trabajo que presentan. En ocasiones se les pide que modifiquen el programa que evalúan y en otros casos se les pide que, en el tiempo disponible, desarrollen partiendo de cero un programa de complejidad similar o menor al que han desarrollado por su cuenta. En esas sesiones de prácticas se detraen 10 minutos para que el alumnado responda un cuestionario con el que se evalúan algunos de los contenidos teóricos que se están estudiando. 


\section{Normas fundamentales \label{sec:fund}} 

\subsection{El cuerpo, la fuente y las líneas}

El \emph{cuerpo} del artículo es una caja de 237~mm de alto y 160~mm
de ancho.  Esta caja está centrada en la hoja, es decir, los márgenes
izquierdo y derecho son de 25~mm a izquierda y derecha y los superior
e inferior son de 31~mm.  Esta caja contendrá el contenido del
artículo: texto, figuras y cuadros\footnote{Cuadro es la palabra
adecuada. \emph{Tabla} es un anglicismo.  Para más detalles,
consúltese el diccionario de la Real Academia
\url{http://www.rae.es/drae/}}, notas a pie de página, etc.  Se ha
dejado espacio fuera del cuerpo para que el maquetador coloque las
cabeceras y números de página.  Esto significa que los autores no
deben preocuparse por esto y que es extremadamente importante no 
salir del cuerpo bajo ninguna circunstancia.

El cuerpo se divide en dos columnas de 76,5~mm de ancho cada una 
separadas por 7~mm. El texto es de fuente Times y de tamaño 10~puntos. 
Para \emph{enfatizar} alguna palabra o frase debe utilizarse la 
cursiva. También debe utilizarse para indicar que alguna palabra está 
en un idioma que no es el español. El texto debe quedar justificado a 
ambos márgenes. En la última página las columnas deben quedar 
equilibradas.
Un inicio de párrafo queda indicado porque la primera palabra está
sangrada\footnote{\emph{Indentado} es otro anglicismo.} 3,5~mm.

\begin{table*}
	\begin{center}
	\begin{tabular}{p{5.2cm}p{4.5cm}cc}
		\textbf{Título} & \textbf{Autores} & \textbf{Edición} & 
		\textbf{Citas (Heterocitas)}\\\hline
		Niveles de Competencia de los Objetivos Formativos en las 
		Ingenierías & Miguel Valero-Garcia y Juan José Navarro & 2001 
		& 21 (18) \\\hline
		Formulación de los Objetivos de una Asignatura en Tres Niveles
		Jerárquicos & Juan José Navarro, Miguel Valero-Garcia, Fermín
		Sánchez Carracedo y Jordi Tubella & 2000 & 18 (8) \\\hline
		¿Cómo serán las asignaturas del EEES? & Fermín Sánchez
		Carracedo & 2005 & 10 (6) \\\hline
		Evaluación continuada a un coste razonable & Miguel
		Valero-Garcia, Luís M. Díaz de Cerio & 2002 & 9 (9) \\\hline
		Hacia la Evaluación Continua Automática de Prácticas de
		Programación & Juan Carlos Rodríguez del Pino, Margarita Díaz
		Roca, Zenón J. Hernández Figueroa y José Daniel González
		Domínguez & 2007 & 8 (7) 
	\end{tabular}
	\end{center}
	\caption{\label{tab:mascit}Artículos más citados en las Jenui.}
\end{table*}

\subsection{Títulos y apartados}

El título de la obra debe estar a 18~puntos, en negrita y centrado.
Los nombres de autores deben estar a 12~puntos, mientras que sus
afiliaciones a tamaño normal (10~puntos).  El título empieza a 10~mm
del borde superior de la caja, es decir, en estas páginas el margen
superior es de 41~mm. Es muy importante dejar este espacio adicional 
ya que el maquetador lo necesita para colocar la información 
editorial del artículo.  La separación entre título y autores es de unas
dos líneas (8~mm).  Esta distancia es la que debe dejarse
aproximadamente entre el final del título y el principio del texto. 

Los títulos de los apartados deben escribirse en 14~puntos y negrita, y 
deben ir numerados. Los de los subapartados deben escribirse a 12~puntos 
en negrita y deben numerarse con número de apartado y 
subapartado. En ambos casos las líneas de título van alineadas a la 
izquierda. Los títulos de apartados y subapartados no pueden acabar 
una columna.

Todos los títulos deben usar mayúsculas sólo si está requerido
gramaticalmente.  Debe ser por tanto ``Estudio de los efectos del
primer curso de programación en Ingeniería Informática'' y no
``Estudio de los Efectos del Primer Curso de Programación en
Ingeniería Informática''.

Desaconsejamos fuertemente el uso de subsubapartados y apartados
menores: una excesiva división del texto hace que el flujo de la prosa
quede muy penalizada y la lectura se hace más dificultosa y
desagradable.  Si no se encuentra una solución mejor, los títulos
deben estar en 10~puntos y negrita y no ir numerados.

\subsection{Extensión}

El número máximo de páginas que se pueden utilizar es de 8~páginas para las ponencias 
y para las descripciones de los recursos docentes, y de 4~páginas para los pósteres 
(en ediciones anteriores, los pósteres estaban limitados a 2~páginas). 
Es imprescindible que los autores utilicen el formato y se ajusten al espacio ya 
desde la primera versión que se someta al proceso de revisión. Lo contrario implicaría 
el rechazo de la contribución.

\section{Listas}

A menudo encontramos listas en las ponencias de las Jenui. 
Cada elemento de la lista va precedido de un símbolo o un número y 
el texto de la lista usa de una línea más estrecha que la del texto. 
Las normas de edición de las listas se muestran a continuación.

\begin{itemize}
	\item Los símbolos que preceden a cada elemento de la lista son 
	números en el caso de listas enumeradas o \emph{balas} 
	($\bullet$) si la lista no está enumerada.
	
	\item La distancia del margen izquierdo del texto de la lista al
	margen izquierdo de la columna (el texto normal) es de 7~mm.  El
	margen derecho de la lista es el mismo que el margen derecho de la
	columna. Es decir, que la anchura de texto de la lista es de 69,5~mm.
	
	\item La distancia del extremo derecho del símbolo al texto de la 
	lista es de 1,5 mm. Los números crecen hacia la izquierda: si hay 
	un número de dos dígitos, el segundo dígito acaba a 1,5 mm del 
	texto de la lista.
	
	\item La distancia adicional del primer elemento de la lista al texto que 
	le precede es aproximadamente de 1,4~mm. Esta también es la distancia adicional del último 
	elemento de la lista al texto que le sucede.
	
	\item Está prohibido tener más de un nivel de 
	listas. Es decir, no se pueden tener listas dentro de listas.
\end{itemize}

\section{Figuras y cuadros}

Las figuras y los cuadros pueden ir dentro de una columna (anchura 
máxima 76,5~mm) o dentro del cuerpo (anchura máxima 160~mm). En ambos 
casos la figura o el cuadro van centrados dentro de la columna o el 
cuerpo. Ambas van numeradas y tienen numeraciones independientes. El 
pie de figura o cuadro empieza con \emph{Figura n:} o \emph{Cuadro 
n:} y usa texto normal a 10~puntos. El pie  va separado 
aproximadamente por la altura de una línea (4~mm) a la figura o 
cuadro y está centrado. La distancia entre el pie y el texto es de 
aproximadamente dos líneas (9~mm).

%\begin{figure}
%\begin{center}
%	\includegraphics[width = \linewidth]{mediaref.pdf}
%\end{center}
%\vspace{-6ex}
%	\caption{\label{fig:medias} Número medio de referencias por 
%	artículo en las 15 ediciones de las Jenui.}
%\end{figure}

%%%%%%%%%%%%%%%%%%%% Fig. %%%%%%%%%%%%%%%%%%%%%%%%%%%%%%%%%%%
\begin{figure}[htbp]
\centerline{\includegraphics[width=\linewidth]{mediaref}}
\caption{\label{fig:medias} Número medio de referencias por artículo en las 15 ediciones de las Jenui} 
\label{fig:mxm_peco} 
\end{figure}
%%%%%%%%%%%%%%%%%%%%%%%%%%%%%%%%%%%%%%%%%%%%%%%%%%%%%%%%%%%%%%

Si el objeto está dentro de una columna puede ir en cualquier lugar 
de la misma, pero si ocupa ambas columnas sólo puede ir en la parte 
superior o inferior de la página.

Es muy importante, y se mirará con mucha atención, que las figuras o 
cuadros no salgan de los márgenes de la columna y del cuerpo.
\begin{figure*}
	\begin{center}
	\includegraphics[width = 0.8\linewidth]{diagrLargo.jpg}
	\end{center}
	\caption{Diagrama en ancho de dos columnas.}
\end{figure*}



\section{Bibliografía}

En el texto se utiliza el número de la referencia, también entre 
corchetes. Hay tres cuestiones a tener en cuenta: (a) El número de 
referencia no puede empezar una línea, hay que poner un ``blanco 
duro'' entre el número de referencia y el texto que lo precede; (b) 
Si hay varias referencias juntas debe usarse un único par de 
corchetes: debe ser ``[2, 7, 14]'' y no ``[2][7][14]''; (c) Si hay 
varias referencias, deben ir en orden numérico ascendente: debe ser  
``[2, 7, 14]'' y no ``[14, 2, 7]''.

El formato de la bibliografía casi merece un artículo propio. 
Recomendamos que las referencias bibliográficas se incluyan en un fichero 
independiente así como el uso del estilo \textit{jenui}. Éste es una adaptación al castellano 
del clásico estilo \textit{plain}.
Si se usa un programa bibliográfico como BibTeX, o EndNote no es necesario 
preocuparse de mucho. En cambio, si se crea la bibliografía a mano 
recomendamos que se mire este artículo o un libro adecuado como 
referencia y que se sigan las siguientes normas:
%
\begin{itemize}
	\item Se debe ser consistente.  Por ejemplo si en una entrada se
	pone ``Actas de las XVII Jornadas de Enseñanza Universitaria de la
	Informática'' en otra no puede ponerse ``Jenui 2012'' y en otra
	``Actas del Jenui'98''.  Si en una entrada el título de un libro
	va en cursiva, en otra entrada no puede ir en letra normal.  No
	importa tanto cómo se escriba sino que siempre se escriba igual.
	
	\item Todo debe ir en español.  Si se importa la bibliografía de
	otra fuente (o se usa BibTeX) es posible que se importen palabras
	en inglés (``J. García \emph{and} P. Quintero, \emph{editors}'').
	Debe cambiarse al español\footnote{Usuarios de BibTeX: editad el
	archivo .bbl justo antes de la última compilación.}. Si se usa el estilo
	\textit{jenui} no es necesario hacer cambios en el nombre de los meses, 
	siempre que se hayan usado macros en la definición del campo mes (por ejemplo, 
	\textit{MONTH = nov,}). Las macros que representan
	los meses del año están formadas siempre por la tres primeras letras del mes en inglés.
	Es decir: jan, feb, mar, apr, $\ldots$ y son correctamente interpretadas por  
	los estilos bibliográficos estándar.
	
	\item Deben indicarse los nombres de todos los autores tal y 
	como lo escriben en el artículo. Si en el artículo los autores 
	aparecen como ``José García Pérez, Pedro Quintero Madariaga y 
	\'Alvaro Nogales Echagüe'' en el texto puede ser adecuado escribir 
	``Tal y como exponen García \emph{et al.}~[5]'' pero en la lista 
	de referencias no puede aparecer ``García \emph{et al.}'' ni 
	siquiera ``J. García, P. Quintero y A. Nogales''. 
	
	\item Un elemento que no tiene ni título ni autor, no pertenece a 
	la bibliografía. En esta categoría entran muchas páginas web y 
	algunos documentos oficiales. Por ejemplo, es mucho mejor 
	escribir ``Podemos encontrar más información en la web de Moodle 
	(\url{http://www.moodle.org})'' que ``Podemos encontrar más 
	información en [11]'' y al ir el lector a la lista de referencias 
	encontrar ``[11] \url{http://www.moodle.org}''. Si la URL es demasiado 
	larga para que quepa cómodamente en una columna, entonces mejor 
	ponerlo como pie de página. 
	
	\item Otro tipo de documento que es mejor no poner como referencia 
	bibliográfica son las leyes. Basta con escribir ``Según la Ley de 
	Reformas Sugestionadas (Ley 7/2012 de 29 de octubre)'': esta 
	información es todo lo que se necesita para encontrarla con 
	facilidad en el BOE.
\end{itemize}

La lista de referencias usa todo el ancho de la columna.  El número 
entre corchetes empieza en el margen de la columna, mientras que el margen
del texto de la referencia está a 9~mm del margen de la columna.


\section{Conclusiones}
Si bien la llegada de los asistentes basados en IA a las aulas no implicará que el profesorado no siga siendo
necesario, sí es cierto que estas tecnologías van a impactar en la praxis docente y hemos de adaptar nuestras
metodologías para incorporar estos cambios.

Algunas preguntas que toda lo expuesto anteriormente motiva y sobre las cuales debemos reflexionar son las
siguientes:
\begin{itemize} 
\item ¿Debemos ignorar la existencia de los asistentes basados en IA o debemos por el contrario, incorporarlos a
    nuestra práctica docente?
\item ¿Hemos de modificar nuestras metodologías docentes?. En caso afirmativo, ¿cómo y qué cambios debiéramos
  introducir?
\end {itemize} 

\balance{}
\bibliographystyle{jenui}
%\nocite{*}
\bibliography{Jenui2023}

\end{document}
